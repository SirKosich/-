\chapter{\label{ch:ch01}ГЛАВА 1: Что такое <<Godot>>} % Нужно сделать главу в содержании заглавными буквами
\textbf{Godot Engine} (читается <<Годо>>)--- это открытый кроссплатформенный 2D и 3D игровой движок под лицензией MIT (это лицензия сбододного программного обеспечения, разработанная Массачусетским технологическим институтом), который разрабатывается сообществом Godot Engine Community. До публичного релиза в виде открытого ПО движок использовался внутри некоторых компаний Латинской Америки. Запускается среда разработки на Android, HTML5, Linux, macOS, Windows, BSD и Haiku, И может экспортировать игровые проекты на Пк, консоли, мобильные и веб-платформы.

\section{\label{sec:ch01/sec01}Раздел 1: Что такое игровые движки и какие есть?}
Игровой движок - это программное обеспечение, которое предоставляет разработчикам инструменты и функционал для создания и разработки видеоигр. Он представляет собой набор библиотек, инструментов, редакторов и других компонентов, которые облегчают процесс создания игр и позволяют разработчикам сосредоточиться на креативной части проекта, не тратя много времени на написание низкоуровневого кода.
Основные компонента игрового движка включает в себя:
\begin{itemize}
    \item Графический движок: Обеспечивает отображение графики, работу с 2D и 3D графикой, анимацией персонажей, освещением, эффектами и другими визуальными аспектами игры.
    \item Физический движок: Позволяет моделировать физические законы в игре, такие как гравитация, столкновения объектов, силы и т.д.
    \item Аудио---движок: Обеспечивает воспроизведение звуков и музыки в игре, управление звуковыми эффектами, микширование звука и другие аспекты аудио-дизайна.
    \item Инструменты разработки: Включают в себя редакторы уровней, ресурсов, скриптов, анимаций, а также другие инструменты для управления проектом и создания контента.
    \item Скриптовые языки:  Многие игровые движки предоставляют возможность программировать игровую логику с помощью скриптовых языков (например, C sharp, JavaScript, Python), что делает процесс разработки более гибким и доступным для широкого круга разработчиков.
\end{itemize}

На данный момент список самых популярных движков выглядит примерно так:
\begin{enumerate}
    \item Unity
    \item Unreal Engine
    \item Godot
    \item CryEngine
    \item GameMaker Studio
\end{enumerate}


\subsection{\label{subsec:ch01/sec01/sub01}Отличие Godot от остальных}
Немного о самом популярном движке Unity, а также его отличие от Godot:
Unity--- тоже кроссплатформенная среда разработки компьютерных игр, разработанная американской компанией Unity Technologies.Основными преимуществами Unity являются наличие визуальной среды разработки, межплатформенной поддержки и модульной системы компонентов. К недостаткам относят появление сложностей при работе с многокомпонентными схемами и затруднения при подключении внешних библиотек.На Unity написаны тысячи игр, приложений, визуализации математических моделей, которые охватывают множество платформ и жанров. При этом Unity используется как крупными разработчиками, так и независимыми студиями.


Основные отличия <<Godot>> и <<Unity>>:
\begin{enumerate}
\item Лицензирование:
    \begin{itemize}
        \item <<Godot>>: Godot распространяется под лицензией MIT, что означает, что его можно использовать бесплатно как для коммерческих, так и для некоммерческих проектов без необходимости платить роялти (это плата, которую разработчики или авторы должны выплачивать владельцам прав на использование их интеллектуальной собственности. В контексте игровой индустрии платить роялти означает, что разработчики должны выплачивать определенный процент от выручки или прибыли от продаж игры или другого продукта владельцу игрового движка, платформы или другому правообладателю).
        \item <<Unity>>: Unity имеет различные версии, включая бесплатную версию (Personal Edition) и платные версии с дополнительными функциями. При этом есть ограничения на доходы, полученные от проекта.
    \end{itemize}
\item Язык программирования:
    \begin{itemize}
        \item <<Godot>>: В Godot используется собственный язык программирования GDScript, который похож на Python. Также поддерживаются C sharp и VisualScript.
        \item <<Unity>>: Unity обычно использует C sharp в качестве основного языка программирования, но также поддерживает JavaScript и Boo.
    \end{itemize}
\item Редактор:
    \begin{itemize}
        \item <<Godot>>: Редактор Godot имеет интуитивно понятный интерфейс и обладает множеством инструментов для создания игр, включая редактор уровней, анимаций, ресурсов и скриптов.
        \item <<Unity>>: Unity также предоставляет мощный редактор с широким спектром возможностей, включая создание игровых объектов, компонентов, анимаций и других элементов.
    \end{itemize}
\item Системы физики:
    \begin{itemize}
        \item <<Godot>>: Godot имеет встроенную физическую систему, которая позволяет моделировать физические законы в игре.
        \item <<Unity>>: Unity также имеет свою физическую систему, которая предоставляет возможности моделирования столкновений, гравитации и других физических эффектов.
    \end{itemize}
\item Сообщество и поддержка:
    \begin{itemize}
        \item <<Godot>>: Сообщество Godot активно развивается и предоставляет множество ресурсов, учебных материалов и документации для разработчиков.
        \item <<Unity>>: Unity имеет огромное сообщество разработчиков, множество плагинов и активную поддержку со стороны компании.
    \end{itemize}
\end{enumerate}
Оба движка имеют свои преимущества и недостатки, и выбор между ними зависит от конкретных потребностей проекта и личных предпочтений разработчика.


\section{\label{sec:ch01/sec02}Чем же Godot и Unity различаются}
Так как мы рассмотрели основные отличия этих двух замечательных движков, логично будет разобрать и их сходства:
\begin{enumerate}
    \item  Мультиплатформенность: Оба движка поддерживают разработку игр для различных платформ, таких как Windows, macOS, Linux, iOS, Android и другие.
    \item Визуальные редакторы:  У обоих движков есть визуальные редакторы, которые позволяют создавать игровые сцены, анимации, интерфейсы и другие элементы без необходимости писать код.
    \item Языки программирования: Оба движка поддерживают несколько языков программирования для разработки игр. Например, Godot использует свой собственный язык GDScript, а Unity поддерживает C sharp, JavaScript и Boo.
    \item Сообщество и документация: У обоих движков активные сообщества разработчиков и обширная документация, что облегчает изучение и использование движков.
    \item Ресурсы и активы: Оба движка имеют магазины активов (Asset Stores), где разработчики могут приобрести готовые ресурсы, такие как модели, текстуры, звуки и другие элементы для использования в своих проектах.
\end{enumerate}


Пример вложенного нумерованного списка:
\begin{enumerate}
\item Первый элемент:
\begin{enumerate}
\item Первый элемент первого элемента;
\item Второй элемент первого элемента;
\end{enumerate}
\item Второй элемент:
\begin{enumerate}
\item Первый элемент второго элемента;
\item Второй элемент второго элемента.
\end{enumerate}
\end{enumerate}

\section{\label{sec:ch01/sec03}Системные требования }

Требования <<Godot>>~\ref{tab:example01}.
\begin{table}[H]
\caption{\centering\label{tab:example01}Системные требования <<Godot>>}
\begin{tabular}{|p{3 cm}|p{5 cm}|}
\hline
Версия операционной системы & Windows 7 SP1+, macOS 10.11+, Linux \\ \hline
Процессор & 64-битный двухъядерный процессор, тактовая частота 2.0 ГГц+ \\ \hline
Графический API & с поддержкой OpenGL ES 3.0+ \\ \hline
Свободное место & 1 ГБ+ \\ \hline
Оперативная память & 2 ГБ \\ \hline
\end{tabular}
\end{table}

Системные требования <<Unity>>~\ref{tab:example02}.
\begin{table}[H]
\caption{\centering\label{tab:example02}Системные требования <<Unity>>}
\begin{tabular}{|p{3 cm}|p{5 cm}|}
\hline
Версия операционной системы & Windows 7 SP1+, 8, 10, macOS 10.12+ \\ \hline
Процессор & поддержка SSE2 \\ \hline
Графический API & с поддержкой DirectX 11 или Metal \\ \hline
Свободное место & 10 GB свободного места \\ \hline
Оперативная память & 4 ГБ \\ \hline
\end{tabular}
\end{table}
\par
Требования, что были указаны выше, являются минимальными для работы с данными движками.